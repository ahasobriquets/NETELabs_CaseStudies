% Template for PLoS
% Version 3.3 June 2016
%
% % % % % % % % % % % % % % % % % % % % % %
%
% -- IMPORTANT NOTE
%
% This template contains comments intended 
% to minimize problems and delays during our production 
% process. Please follow the template instructions
% whenever possible.
%
% % % % % % % % % % % % % % % % % % % % % % % %

% Created to hold supplementary information

\documentclass[10pt,letterpaper]{article}
\usepackage[top=0.85in,left=2in,footskip=0.75in]{geometry}

% amsmath and amssymb packages, useful for mathematical formulas and symbols
\usepackage{amsmath,amssymb}

% Use adjustwidth environment to exceed column width (see example table in text)
\usepackage{changepage}

% Use Unicode characters when possible
\usepackage[utf8x]{inputenc}

% textcomp package and marvosym package for additional characters
\usepackage{textcomp,marvosym}

% cite package, to clean up citations in the main text. Do not remove.
\usepackage{cite}

% Use nameref to cite supporting information files (see Supporting Information section for more info)
\usepackage{nameref,hyperref}

% line numbers
\usepackage[right]{lineno}

% ligatures disabled
\usepackage{microtype}
\DisableLigatures[f]{encoding = *, family = * }

% color can be used to apply background shading to table cells only
\usepackage[table]{xcolor}

% array package and thick rules for tables
\usepackage{array}

% create "+" rule type for thick vertical lines
\newcolumntype{+}{!{\vrule width 2pt}}

% create \thickcline for thick horizontal lines of variable length
\newlength\savedwidth
\newcommand\thickcline[1]{%
  \noalign{\global\savedwidth\arrayrulewidth\global\arrayrulewidth 2pt}%
  \cline{#1}%
  \noalign{\vskip\arrayrulewidth}%
  \noalign{\global\arrayrulewidth\savedwidth}%
}

% \thickhline command for thick horizontal lines that span the table
\newcommand\thickhline{\noalign{\global\savedwidth\arrayrulewidth\global\arrayrulewidth 2pt}%
\hline
\noalign{\global\arrayrulewidth\savedwidth}}


% Remove comment for double spacing
%\usepackage{setspace} 
%\doublespacing

% Text layout
\raggedright
\setlength{\parindent}{0.5cm}
\textwidth 5.25in 
\textheight 8.75in

% Bold the 'Figure #' in the caption and separate it from the title/caption with a period
% Captions will be left justified
\usepackage[aboveskip=1pt,labelfont=bf,labelsep=period,justification=raggedright,singlelinecheck=off]{caption}
\renewcommand{\figurename}{Fig}

% Use the PLoS provided BiBTeX style
\bibliographystyle{plos2015}

% Remove brackets from numbering in List of References
\makeatletter
\renewcommand{\@biblabel}[1]{\quad#1.}
\makeatother

% Leave date blank
\date{}

% Header and Footer with logo
\usepackage{lastpage,fancyhdr,graphicx}
\usepackage{epstopdf}
\pagestyle{myheadings}
\pagestyle{fancy}
\fancyhf{}
\setlength{\headheight}{27.023pt}
\lhead{\includegraphics[width=3.0in]{PLOS-submission.eps}}
\rfoot{\thepage/\pageref{LastPage}}
\renewcommand{\footrule}{\hrule height 2pt \vspace{2mm}}
\fancyheadoffset[L]{1.25in}
\fancyfootoffset[L]{1.25in}
\lfoot{\sf PLOS}

%% Include all macros below

\newcommand{\lorem}{{\bf LOREM}}
\newcommand{\ipsum}{{\bf IPSUM}}

%% END MACROS SECTION

\begin{document}
\vspace*{0.2in}

\begin{table}[!ht]
\centering
\caption{
{\bf Elite Performers}}
\vspace{2.5 mm}
\begin{tabular}{|l| l| l|}
\hline
Name& nPIR & PPR \\
\hline
\hline
Ferrara N.	 & 46693 & 1.12 \\ 
Folkman J. & 23660 & 1.15 \\ 
Ullrich A. & 23034 & 1.46 \\ 
Jain R. & 15267 & 1.13 \\ 
Heldin C.	& 15148 & 1.21 \\ 
Druker B.	& 15088 & 1.16 \\ 
Schlessinger J.	& 14996 & 1.48 \\ 
Dvorak H.	 & 14230 & 1.13 \\ 
 Alitalo K.	& 13812 & 1.24 \\ 
 Slamon D. & 13587 & 1.46 \\ 
 Baselga J. & 12908 & 1.36 \\ 
 Kantarjian H. & 12029 & 1.17 \\ 
 Hicklin D.	 & 11775 & 1.17 \\ 
 Witte O. & 11449 & 1.08 \\ 
 Hanahan D. & 11072 & 1.19 \\ 
 Buchdunger E.	 & 11032 & 1.22 \\ 
 Risau W.	& 10950 & 1.24 \\ 
 Talpaz M.	 & 10713 & 1.13 \\ 
 Mendelsohn J.	 & 10534 & 1.54 \\ 
 Lydon N. & 9988 & 1.18 \\ 
 Goldman J. & 9927 & 1.11 \\ 
 Shibuya M. & 9639 & 1.21 \\ 
 Kitamura Y. & 9486 & 1.24 \\ 
 Waldmann H. & 9363 & 1.04 \\ 
 Kerbel R.	 & 9266 & 1.16 \\ 
   \hline
\end{tabular}
\vspace{2.5 mm}
\begin{flushleft} Researchers with the highest nPIR scores in the network of 5 anti-cancer therapeutics are listed. Also shown for each researcher is their PIRpartitionRatio (PPR). The nPIR indicates influence across all five networks and the PPR provides an estimate of how this influence is partitioned across each of the five networks (Materials and Methods). This list and its ordering should be considered in the context of the data being analyzed and not interpreted as an absolute ordering of research excellence in the field.
\end{flushleft}
\label{table2}
\end{table}

\end{document}

